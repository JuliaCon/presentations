\documentclass[compressed,dvips,letter]{beamer}

\mode<presentation>
{
  \usetheme{Boadilla}
  \setbeamercovered{dynamic}
}

%****************************************************************************
%
%****************************************************************************
%
\usepackage{graphicx}
% \usepackage{ifthen}
\usepackage{fancybox}
\usepackage{textpos}
% \usepackage{color}
% \usepackage{hyphenat}
% \renewcommand{\sfdefault}{pag}  % Use a nicer font for the headers
% \usepackage{float}
% \usepackage{paralist}
\usepackage{url}
% \usepackage{moreverb}
% \usepackage{longtable}

\usepackage{bm,bbm}
\usepackage{amsfonts,amsmath,amsthm,amstext,amssymb}
\usepackage{latexsym}
\usepackage[usenames]{pstricks}
%\usepackage{pst-plot,pst-3dplot,pst-node}
\usepackage{psfrag}
\usepackage{enumerate}
\usepackage[sans]{dsfont}
\usepackage[mathscr]{euscript}
\usepackage{mathtools}
%\usepackage{cite}
\usepackage{multido}

\graphicspath{{figs/}}

\DeclareMathOperator{\rank}{rank}      
\DeclareMathOperator{\lexmin}{lexmin}
\DeclareMathOperator{\arglexmin}{arglexmin}
\DeclareMathOperator{\argmin}{argmin}
\DeclareMathOperator{\relint}{relint}                                % relative interior
\DeclareMathOperator{\inter}{int}                                    % interior of a set
\DeclareMathOperator{\gph}{gph}                                      % graph of a mapping
\DeclareMathOperator{\epi}{epi}                                      % epigraph of a function
\DeclareMathOperator{\dom}{dom}                                      % domain of a function
\DeclareMathOperator{\bnd}{bnd}                                      % boundary of a set
\DeclareMathOperator{\aff}{affh}                                     % affine hul
\DeclareMathOperator{\co}{convh}                                     % convex hull
\DeclareMathOperator{\clco}{\overline{convh}}                        % closed convex hull

\newrgbcolor{orange}{1.00 0.65 0.00}
\newrgbcolor{mellow}{0.847 0.72 0.525}
\newrgbcolor{grey}{0.5 0.5 0.5}
\newrgbcolor{darkgrey}{0.7 0.7 0.7}
\newrgbcolor{lightgrey}{0.3 0.3 0.3}
\newrgbcolor{verylightgrey}{0.9 0.9 0.9}
\newrgbcolor{darkred}{0.5 0 0}
\newrgbcolor{darkgreen}{0 0.5 0}
\newrgbcolor{darkblue}{0 0 0.5}
\newrgbcolor{copper}{1 0.62 0.4}
\newrgbcolor{aquamarine}{0.49 1 0.83}
\newrgbcolor{pink}{1 0.4 0.6}
\newrgbcolor{lightblue}{.30 .86 .89}
\newrgbcolor{highlight}{1 1 0.9}
\newrgbcolor{periwinkle}{0.8 0.8 1}
\newrgbcolor{indigo}{0.4 0 1}
\newrgbcolor{babyblue}{0.43 1 1}
\newrgbcolor{royalblue}{0.031 0.298 0.619}
\newrgbcolor{navyblue}{0 0 0.501}
\newrgbcolor{sapphire}{0.031 0.145 0.403}
\newrgbcolor{denim}{0.082 0.376 0.741}
\newrgbcolor{cobaltblue}{0 0.278 0.67}
\newrgbcolor{ceruleanblue}{0.164 0.321 0.745}

\def\mc{\mathcal}
\def\mbf#1{\mathbf{#1}}
\def\mbb#1{\mathbb{#1}}
\def\bb{(\cdot)}
\def\bbb{(\cdot,\cdot)}
\def\bbbb{(\cdot,\cdot,\cdot)}
\def\eqbyd{:=}                                                      %equal by definition
\def\st{\mbox{ such that }}
\def\R{\mathds{R}}                                                  % real numbers
\def\Rinf{\bar{\R}}                                                 % R U +-\infty
\def\Rnm#1#2{\R^{#1\times#2}}                                         % for matrices
\def\N{\mathds{N}}                                                  % N = {1,2,...}
\def\powset#1{2^{#1}}                                               % power set
\def\proj#1#2{\mathrm{Proj}_{#1}\left(#2\right)}
\def\BackSet#1{\mathscr{B}\left(#1\right)}
\def\Bmaxmin#1{\mathscr{B}_{\small\mbox{max-min}}\left(#1\right)}
\def\Bminmax#1{\mathscr{B}_{\small\mbox{min-max}}\left(#1\right)}
\def\Smaxmin#1{\mathscr{S}_{\small\mbox{max-min}}\left(#1\right)}
\def\Sminmax#1{\mathscr{S}_{\small\mbox{min-max}}\left(#1\right)}
\def\restrict#1#2{#1 \mid #2}
\def\row#1#2{#1_{\left[#2,\cdot\right]}}                            % row of a matrix
\def\col#1#2{#1_{\left[#2\right]}}                                  % column of a vector
\def\tto{\rightrightarrows}                                         % double arrow for set-valued maps
\def\dash{\nobreakdash-\hspace{0pt}}
\def\infsup{\textrm{inf}\dash\textrm{sup} }
\def\supinf{\textrm{sup}\dash\textrm{inf} }
\def\minmax{\textrm{min}\dash\textrm{max} }
\def\maxmin{\textrm{max}\dash\textrm{min} }                       
\def\dprod#1{\langle{#1}\rangle}                                    % dot product
\def\CI#1#2{\mathcal{I}_c(#1,#2)}                                   % controllability index
\def\g#1{g_{\mbox{\tiny $#1$}}}                                     % gauge function
\def\rb#1{{\color{royalblue}{#1}}}
\def\bcite#1{\rb{\small #1}}
%

%
% Beamer setup
%
\mode<presentation>
{
  \usetheme{Boadilla}
  \setbeamercovered{dynamic}
}

\setbeamertemplate{navigation symbols}{}
\setlength{\TPHorizModule}{1cm}
\setlength{\TPVertModule}{1cm}

\usefonttheme{professionalfonts}

\usepackage{array}
\usepackage{subfigure}
\usepackage{multirow}
\usepackage{arydshln}
\usepackage{verbatim}
\usepackage{graphicx}
\usepackage{amsmath}
%\usepackage{url}
\usepackage{hyperref}
\usepackage{breakurl}
%\usepackage{pdfpages}


\setbeamertemplate{navigation symbols}{}
\setlength{\TPHorizModule}{1cm}
\setlength{\TPVertModule}{1cm}


\usefonttheme{professionalfonts}
\title[How to Support Windows]
{How to \only<2->{Support }Windows\\
\onslide<2->{\vspace{5pt}
\small Cross-platform installation and testing for Julia packages}}

\author{Tony Kelman \vspace{-10pt}}

\institute[{\fontfamily{pcr}\selectfont @tkelman}]
{
  \footnotesize {\fontfamily{pcr}\selectfont @tkelman}
}
\date[JuliaCon, June 25 2015]{\includegraphics[width=85pt]{fig/julia_logo.eps}
\\JuliaCon\\June 25, 2015}

\subject{}


\begin{document}

%\setbeamertemplate{blocks}[rounded][shadow=true]

\setbeamertemplate{footline}{}
\begin{frame}
  \titlepage
\end{frame}


%
%===========================================================================================
%
\setcounter{framenumber}{0}
\setbeamertemplate{footline}[infolines theme]



\begin{frame}[fragile]\frametitle{Reasons to care about Windows support}
\begin{itemize}
  \item Education
  \begin{itemize}
    \item Not all students can afford a Mac
    \item Not everyone knows how to use Linux
  \end{itemize}
  \item Some industries
  \begin{itemize}
    \item Require proprietary Windows-only software
    \item Conservative IT policies
    \item Instrumentation and embedded hardware
  \end{itemize}
  \onslide<2->{
  \item Masochism
  \begin{itemize}
    \item Fun to find and fix the bugs no one else wants to touch
  \end{itemize}
  }
\end{itemize}

\end{frame}
%
%--------------------------------------------------------------
%



\begin{frame}[fragile]\frametitle{Good news and bad news}
\begin{itemize}
  \item Biggest question for package authors - binary dependencies
  \item Is your package code pure Julia?
  \begin{itemize}
    \item Base Julia is portable enough that most code will work
    \item Few things to watch out for, shelling out, filesystem differences
    \item Run your tests on platforms you don't use
  \end{itemize}
  \item Are you wrapping a \texttt{C}, \texttt{C++}, or Fortran library?
  \begin{itemize}
    \item Harder problem, but not hopeless
    \item Automated tools for building and distributing binaries
  \end{itemize}
\end{itemize}

\end{frame}
%
%--------------------------------------------------------------
%



\begin{frame}[fragile]\frametitle{Continuous integration for Julia packages on Windows}
\begin{center}
\includegraphics[width=120pt]{fig/appveyor-logo-with-text.eps}
\end{center}
\begin{itemize}
  \item AppVeyor {\small \url{http://www.appveyor.com}} - free for open source projects
  \item Same idea as Travis CI {\small \url{https://travis-ci.org}}, but for Windows
  \item Add \texttt{appveyor.yml} to your repository and click one button to enable
  \begin{itemize}
    \item Starts up a Windows VM on every commit and pull request
    \item Installs your package, runs your tests, reports status back to GitHub
  \end{itemize}
  \item Template \texttt{appveyor.yml} file in Example.jl package \\ \url{https://github.com/JuliaLang/Example.jl/blob/master/appveyor.yml}
\end{itemize}

\end{frame}
%
%--------------------------------------------------------------
%




\begin{frame}[fragile]\frametitle{Example appveyor.yml file}
{\scriptsize
\begin{verbatim}
environment:
  matrix:
  - JULIAVERSION: "julialang/bin/winnt/x86/0.3/julia-0.3-latest-win32.exe"
  - JULIAVERSION: "julialang/bin/winnt/x64/0.3/julia-0.3-latest-win64.exe"
  - JULIAVERSION: "julianightlies/bin/winnt/x86/julia-latest-win32.exe"
  - JULIAVERSION: "julianightlies/bin/winnt/x64/julia-latest-win64.exe"

install:
# Download most recent Julia Windows binary
  - ps: (new-object net.webclient).DownloadFile(
        $("http://s3.amazonaws.com/"+$env:JULIAVERSION),
        "C:\projects\julia-binary.exe")
# Run installer silently, output to C:\projects\julia
  - C:\projects\julia-binary.exe /S /D=C:\projects\julia

build_script:
# Need to convert from shallow to complete for Pkg.clone to work
  - IF EXIST .git\shallow (git fetch --unshallow)
  - C:\projects\julia\bin\julia -e "versioninfo();
      Pkg.clone(pwd(), \"Example\"); Pkg.build(\"Example\")"

test_script:
  - C:\projects\julia\bin\julia --check-bounds=yes -e "Pkg.test(\"Example\")"
\end{verbatim}
}
\end{frame}
%
%--------------------------------------------------------------
%


\begin{frame}[fragile]\frametitle{Binary dependencies, the hard part}

\begin{itemize}
  \item What makes building scientific software on Windows hard?
  \item Windows is not POSIX - no \texttt{sh}, no coreutils
  \item Win32 system API, \texttt{NtQueryInformationFile} and other unspeakable horrors
  \item No built-in package management for native dependency handling
  \onslide<2->{
  \item Let's talk about compilers
  \begin{itemize}
    \item Visual Studio, the platform native choice
    \begin{itemize}
      \item World-class debugger and IDE
      \item \texttt{C} is an afterthought, very recently started supporting C99 (mostly)
      \item No Fortran compiler, no 64-bit inline assembly
    \end{itemize}
    \item Intel's compilers, the expensive option
    \begin{itemize}
      \item High performance \texttt{C}, \texttt{C++}, and Fortran compilers
      \item Has features missing from MSVC, but not free to install
    \end{itemize}
  \end{itemize}
  }
\end{itemize}
\end{frame}
%
%--------------------------------------------------------------
%


\begin{frame}[fragile]\frametitle{Compilers and build systems, cont'd}

\begin{itemize}
  \item Open-source compilers
  \begin{itemize}
    \item Clang, the newcomer
    \begin{itemize}
      \item Windows \texttt{C++} exception handling a work in progress
      \item No Fortran compiler
    \end{itemize}
    \item GCC (MinGW-w64), yes it does run everywhere
    \begin{itemize}
      \item \texttt{gfortran} only option for open source scientific community
      \item Excellent cross compilation support
    \end{itemize}
  \end{itemize}
  \onslide<2->{
  \item Let's talk about build systems
  \begin{itemize}
    \item Autotools \texttt{./configure} assumes a POSIX shell
    \item Virtually all Makefiles written for GNU \texttt{make}
    \item CMake an emerging standard, not universally used yet
  \end{itemize}
  \item Best way to build scientific libraries for Windows today?
  \begin{itemize}
    \item Build with GCC using a POSIX environment like Cygwin, or MSYS2, \\
    or cross compile from Linux
    \item Distribute just the binaries, Julia packages only need \texttt{dll} files
    \begin{itemize}
      \item Users do not need compilers or build environment at install or run time
    \end{itemize}
  \end{itemize}
  }
\end{itemize}
\end{frame}
%
%--------------------------------------------------------------
%



\begin{frame}[fragile]\frametitle{Building and providing Windows binaries}

\begin{itemize}
  \item Small, self-contained research code?
  \begin{itemize}
    \item Manually compile and upload a \texttt{dll}
    \item From Linux you can use \texttt{x86\_64-w64-mingw32-gcc} to cross compile
    \item Julia package just downloads compiled \texttt{dll} with \texttt{Binaries} provider from BinDeps.jl \url{https://github.com/JuliaLang/BinDeps.jl}
    \item Add the following to a file \texttt{deps/build.jl} in your package
  \end{itemize}
\end{itemize}
{\scriptsize
\begin{verbatim}
using BinDeps

@BinDeps.setup

libfoo = library_dependency("libfoo")

# Sources and BuildProcess providers for source build on Linux
# Homebrew.jl provider for binaries on OS X

provides(Binaries, URI(string("https://github.com/foo/libfoo/",
    "releases/download/v#.#.#/libfoo-#.#.#.tar.gz")),
    [libfoo], os = :Windows)

@BinDeps.install Dict([(:libfoo, :libfoo)])
\end{verbatim}
}
\end{frame}
%
%--------------------------------------------------------------
%



\begin{frame}[fragile]\frametitle{openSUSE build service for cross compiling}

\begin{itemize}
  \item Established library with complicated dependencies?
  \begin{itemize}
    \item openSUSE has many cross-compiled Windows packages \url{https://build.opensuse.org/project/show/windows:mingw:win64}
    \item Easy to use automated build service, a little like GitHub
    \item Upload source tarball, write a \texttt{spec} file with compilation instructions
  \end{itemize}
\end{itemize}
{\tiny
\begin{verbatim}
%define _basename zeromq
Name:           mingw64-%{_basename}
Version:        4.0.5
Release:        0
#!BuildIgnore: post-build-checks
Summary:        Lightweight messaging kernel
License:        LGPL-3.0+
Group:          Productivity/Networking/Web/Servers
Url:            http://www.zeromq.org/
Source:         http://download.zeromq.org/%{_basename}-%{version}.tar.gz
BuildRequires:  mingw64-cross-binutils
BuildRequires:  mingw64-cross-gcc
BuildRequires:  mingw64-cross-gcc-c++
BuildRequires:  mingw64-cross-pkg-config
BuildRequires:  mingw64-filesystem
BuildRoot:      %{_tmppath}/%{name}-%{version}-build
%_mingw64_package_header_debug
BuildArch:      noarch
\end{verbatim}
}
\end{frame}
%
%--------------------------------------------------------------
%



\begin{frame}[fragile]\frametitle{spec file for cross compiling}

\begin{minipage}{0.44\textwidth}
{\tiny
\begin{verbatim}
%package devel
Summary:        Development files for ZeroMQ
Group:          Development/Languages/C and C++
Requires:       %name = %version
Provides:       mingw64-libzmq-devel = %{version}

%_mingw64_debug_package

%prep
%setup -q -n %{_basename}-%{version}

%build
%{_mingw64_configure}
%{_mingw64_make} %{?_smp_mflags} V=1

%install
%{_mingw64_make} DESTDIR=%{buildroot} install

%files
%defattr(-,root,root,-)
%doc COPYING COPYING.LESSER
%{_mingw64_bindir}/libzmq.dll
%{_mingw64_bindir}/curve_keygen.exe

%files devel
%defattr(-,root,root,-)
%doc AUTHORS ChangeLog COPYING COPYING.LESSER NEWS
%{_mingw64_includedir}/zmq*
%{_mingw64_libdir}/libzmq.dll.a
%{_mingw64_libdir}/pkgconfig/libzmq.pc

%changelog
\end{verbatim}
}
\end{minipage}
\begin{minipage}{0.55\textwidth}
{\footnotesize
$\longleftarrow \texttt{./configure --host=x86\_64-w64-mingw32}$

$\longleftarrow \texttt{make}$

\vspace{8pt}

$\longleftarrow \texttt{make install}$

\vspace{25pt}}
\end{minipage}
\end{frame}
%
%--------------------------------------------------------------
%



\begin{frame}[fragile]\frametitle{WinRPM.jl for library installation}

\begin{itemize}
  \item WinRPM.jl \url{https://github.com/JuliaLang/WinRPM.jl} \\
  primarily written by {\fontfamily{pcr}\selectfont @vtjnash} and {\fontfamily{pcr}\selectfont @ihnorton}
  \item Parses RPM metadata, dependencies and latest versions, \\
  from openSUSE build service and downloads binaries
  \item Add the following to \texttt{deps/build.jl} in your package
\end{itemize}
{\scriptsize
\begin{verbatim}
using BinDeps

@BinDeps.setup

libfoo = library_dependency("libfoo")

# other providers for Linux, OS X

@windows_only begin
    using WinRPM
    provides(WinRPM.RPM, "foo", [libfoo], os = :Windows)
    # replace "foo" with "Name:" value from opensuse package
    # but without the "mingw64-" prefix
end

@BinDeps.install Dict([(:libfoo, :libfoo)])
\end{verbatim}
}
\end{frame}
%
%--------------------------------------------------------------
%


\begin{frame}[fragile]\frametitle{Not an impossible problem}

\begin{itemize}
  \item A simple user experience can be achieved
  \item Packages can work on Windows even if developers don't use it
  \item Provide binaries, turn on automated testing
\end{itemize}

\vspace{20pt}

\center
Questions?

\end{frame}
%
%--------------------------------------------------------------
%



\end{document}
